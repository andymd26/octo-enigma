\documentclass[10pt]{amsart} 
\usepackage{graphicx} 
\usepackage{float} % necessary for placement of figures
\usepackage{amsmath}
\usepackage{tabularx}
\usepackage{gensymb}
\usepackage[style = authoryear, sorting = nyt, backend = biber]{biblatex}
\bibliography{library.bib}

\title{Where to incorporate climate information?}
\author{Andrew Blohm}
\date{\today}

\begin{document}
\maketitle
\section{Where to incorporate climate information?}
It is an open research question of how to improve the resilience and reduce the vulnerability of human and natural systems to climate change.
In particular, how best to build in climate information? 
Agencies are now required as a result of Executive Order [?] to determine the vulnerability of agency assets to climate change.
For an organization as large as the Department of Defense (DoD) this is a large undertaking. 
In previous work we put forward the idea that conducting vulnerability assessments at each site is inefficient, as the threats of climate change and more importantly the mission impacts are non-homogenously distributed across space.
Nevertheless in this work we didn't identify the optimal means of creating change throughout the organization (i.e., improved resilience, reduced vulnerability).

We have conducted vulnerability analyses across Department of Defense and Department of Energy sites and have begun working towards best practices both in methods for conducting vulnerability analyses and the creation of toolboxes to match with site characteristics.
One issue that always comes up is where to incorporate climate information so as to ensure resilience is maintained for future periods as well.
We find that existing processes are the best spot to incorporate climate information, as compared to generating new processes. 
Existing processes have stakeholders, methods, user communities, etc.
All of which would need to be established for any new process.
The DoD has a number of required products that could benefit from the inclusion of climate projections such as the infrastructure standards through processes such as the Unified Facilities Criteria (UFCs), installation master planning, Integrated Natural Resources Management Plan (INRMP), emergency management planning, etc.\footnote{UFCs are updated on a recurring basis with the frequency of updates dependent upon the application (i.e., building codes every five years versus hyperbaric facilities updated every 20 years)[Conversation with Aaron Philips].}

Each of these documents and/or processes has an established user community.
At the moment UFCs might make mention of the need to plan or incorporate climate change but lacks sufficient detail on how this should happen.
From our conversations it became apparent that the structural engineers just need guidance. 
Existing planning processes mean that by the time it gets to the design desk it is too late to incorporate climate change \parencite{}[conversation with Aaron Philips].
Further existing hurdles are exacerbating the problem.
For example, it is presently easier to repair using operations and repair money as compared to replacement using milcon money \parencite{}[conversation with John Hall].

We spent a significant amount of time interviewing stakeholders at sites across the DoD portfolio to better understand existing decision making processes; both in the operation of existing assets and construction of new assets.
We find that many sites have a poor understanding of the impacts of current climate on operations, which limits the ability to incorporate climate projections into decision-making processes.
We suggest a first step be improved data collection at each site on the variables that lead to mission impacts.
For example, we have found in the training and test range community that cancellations of training or test events is tracked but the cause is not.
There is an attribution issue, which means that installation commanders and central command personnel cannot track changes in the ability of these installations to meet mission objectives as a direct or indirect result of changes in climate.
Further, commanders cannot optimize the timing of their training and tests to better reflect the ongoing changes in their climate. 
Finally, not collecting this information prevents the identification of where their own experiences reside in regards to expectations from scenarios (i.e., faster, slower than under scenarios), which impacts the ability to plan.

The structure of the Department of Defense (DoD) would seem to epitomize an organization capable of meeting the challenges associated with developing, communicating, and adopting vulnerability assessments and adaptation alternatives for each of its sites in its large portfolio.
The military is hierarchical with command and control structures, which should be capable of implementing directives from central command at each site. 
However, in our case studies we find that the participatory process might be more important in the DoD than in other organizations because of the challenges that exist.

First, non-civilian members of the DoD are highly transient often changing duty stations.
As a result, the institutional memory is often found in civilian contractors.
The views of these civilian contractors on climate change determines much of what happens next.
Second, existing budget processes do not value projects that reduce vulnerability to climate change.
Third, existing budget processes (especially for large assets) are hard to change once they've begun and thus, even in the case where new information is received it might be impossible to adjust course. 

[Insert discussion of partners and proceses that we worked with]
Existing structures in the test and range communities.
The test range community meets regularly, and reports to a range commanders council.
Engineering Senior Executive Panel (ESEP) reports to Coordinating Panel \parencite{}[conversation with Aaron Philips and John Hall]

\section{Unified Facilities Criteria}
The Unified Facilities Criteria (UFC) provide guidance for developing and maintaining unified facilities design and construction for planning, design, construction, sustainment, restoration, and modernization of DoD facilities \parencite{}(\url{https://www.wbdg.org/ccb/browse_cat.php?c=4}).
The United States Army Corps of Engineers (HQUSACE), Naval Facilities Engineering Command (NAVFAC), and the Air Force Civil Engineer Center (AFCEC) administer the UFC system within their own departments \parencite{} (\url{https://www.wbdg.org/ccb/browse_cat.php?c=4}).
UFCs intended for all participating agencies have a document number that does not end in an alphabetical letter. 
Otherwise, a document with an 'A', 'N', or 'F' at the end of the document corresponds to guidance intended solely for the USACE, NAVFAC, or AFCEC, respectively \parencite{}(\url{https://www.wbdg.org/ccb/browse_cat.php?c=4}).
UFCs are organized into four series: (1) Policy, Procedures, and Guidance; (2) Master Planning; (3) Discipline-specific Criteria; and, (4) Multi-disciplinary and Facility-Specific Design. 
A complete listing of the current UFCs can be found at \url{https://www.wbdg.org/ccb/browse_cat.php?c=4}.

The UFCs are a good program to use as a platform for resilience building for several reasons.
First, UFCs are an existing program and would not require the creation of a new program, training users, and other associated costs of such an effort.
Second, given its use throughout the services, small changes in the UFCs could lead to large changes in resilience.
Third, there are multiple points in the UFC program where climate information could be integrated.
Finally, the program directly ties into the systems that we care about. 

Of the complete set of UFCs, we performed a cursory analysis to identify UFCs that would benefit from the integration of climate information, as well as climate projections.

Series 1: Policy, Procedures, and Guidance
\begin{enumerate}
\item UFC 1-200-02 High Performance and Sustainable Building Requirements, with Change 3 (03-01-2013)
\item UFC 1-201-01 Non-Permanent DoD facilities in Support of Military Operations (01-01-2013)
\item UFC 1-201-02 Assessment of existing facilities for use in Military Operations (06-01-2014)
\item UFC 1-202-01 Host Nation Facilities in Support of Military Operations (09-01-2013)
\item UFC 1-300-07A Design Build Technical Requirements (03-01-2005)
\item UFC 1-300-09N Navy and Marine Corps Design Procedures, with Change 2 (05-01-2014)
\end{enumerate}

Series 2: Master Planning
\begin{enumerate}
\item UFC 2-000-05N (formerly P-80) Facility Planning Criteria for Navy/Marine Corps Shore Installations
\item UFC 2-100-01 Installation Master Planning (05-15-2012)
\end{enumerate}

Series 3: Discipline-specific criteria
\begin{enumerate}
\item UFC 3-201-01 Civil Engineering (-06-01-2013)
\item UFC 3-220-05 Dewatering and Groundwater Control (01-16-2004)
\item UFC 3-320-01 Water storage, Distribution, and Transmission, with Change 2 (11-01-2012)
\item UFC 3-230-03 Water Treatment (11-01-2012)
\item UFC 3-230-06A Subsurface Drainage, with Changes 1-2 (01-16-2004)
\item UFC 3-240-01 Wastewater Collection, with Change 1 (11-01-2012)
\item UFC 3-240-02 Domestic Wastewater Treatment (11-01-2012)
\item UFC 3-260-01 Airfield and Heliport Planning and Design (11-17-2008)
\item FC 3-260-06F Air Force Design, Construction, Maintenance, and Evaluation of Snow and Ice Airfields in Antarctica (06-01-2015)
\item UFC 3-301-01 Structural Engineering, with Change 1 (06-01-2013)
\item UFC 3-400-02 Design: Engineering Weather Data (02-28-2003)
\item UFC 3-401-01 Mechanical Engineering, with Change 1 (07-01-2013)
\item UFC 3-410-01 Heating, Ventilating, and Air Conditioning Systems, with Change 2 (07-01-2013)
\item UFC 3-810-01N Navy and Marine Corps Environmental Engineering for Facility Construction (03-01-2016)
\end{enumerate}

Series 4: Multi-disciplinary and facility-specific design
\begin{enumerate}
\item UFC 4-141-10N Design: Aviation Operation and Support Facilities (01-16-2004)
\item UFC 4-150-02 Dockside Utilities for Ship Service, with Change 5 (05-12-2003)
\item UFC 4-150-06 Military Harbors and Coastal Facilities, with Change 1 (12-12-2001)
\item UFC 4-150-07 Maintenance and Operation: Maintenance of Waterfront Facilities, with Change 1 (06-19-2001)
\item UFC 4-151-10 General Criteria for Waterfront Construction, with Change 1 (09-10-2001) 	
\item UFC 4-152-01 Design: Piers and Wharves; with Change 1 (07-28-2005)
\item UFC 4-152-07 Design: Small Craft Berthing Facilities; with Change 1 (07-14-2009)
\item UFC 4-159-03 Design: Moorings, with Change 1 (10-03-2005)
\item UFC 4-171-01N Design: Aviation Training Facilities (01-16-2004)
\item UFC 4-213-10 Design: Graving Drydocks, with Change 1 (08-15-2002)
\item UFC 4-213-12 Drydocking Facilities Characteristics (06-19-2003)
\end{enumerate}

\section{Guidance documents: Sea Level Rise}
These documents represent the extent of guidance documents for the military services on incorporating sea level rise into existing planning mechanisms, operations, etc.

\begin{enumerate}
\item ER 1100-2-8160: Policies for Referencing Project Elevation Grade to Nationwide Vertical Datums (03-01-2009)
\item ER 1100-2-8162: Incorporating Sea Level Change in Civil Works Program (12-31-2013)
\item ETL 1100-2-1: Procedures to Evaluate Sea Level Change (06-30-2014)
\item ER 1105-2-100: Planning Guidance Notebook (04-2000) \url{http://www.publications.usace.army.mil/USACEPublications/EngineerRegulations/tabid/16441/u43546q/313130352D322D313030/Default.aspx}
\item EM 1110-2-6056: Standards and Procedures for Referencing Project Elevation Grades to Nationwide Vertical Datums. \url{http://www.publications.usace.army.mil/USACEPublications/EngineerManuals/tabid/16439/u43544q/313131302D322D36303536/Default.aspx}
\item ECB 2016-5: Using Non-NOAA Tide Gauge Records for Computing Relative Sea Level Change (01-27-2016)
\item National Research Council (1987) Responding to Changes in Sea Level: Engineering Implications. Washington, DC: National Academy Press. \url{http://www.nap.edu/catalog.php?record_id=1006}
\item National Research Council (2012) Sea-Level Rise for the Coasts of California, Oregon, and Washington: Past, Present, and Future. Committee on Sea Level Rise on California, Oregon, and Washington, Board on Earth Sciences and Resources and Ocean Studies Board. Washington, DC: National Academy Press.
\item Intergovernmental Oceanographic Commission (1985) Manual on Sea Level Measurement and Interpretation, Volume I. Intergovernmental Oceanographic Commission Manuals and Guides-14. \url{http://unesdoc.unesco.org/images/0006/000650/065061eb.pdf}
\item Intergovernmental Oceanographic Commission (2012) Manual on Sea-Level Measurements and Interpretation. Volume 4 — An Update to 2006 (T. Aarup, M. Merrifield, B. Perez, I. Vassie, and P. Woodworth, eds.). IOC Manuals and Guides No. 14, vol. IV; JCOMM Technical Report No. 31; WMO/TD. No. 1339. Paris, France: Intergovernmental Oceanographic Commission.
\item C. Zervas, S. K. Gill, and W. Sweet (2013) Estimating Vertical Land Motion from Long-Term Tide Gauge Records. Technical Report. Silver Spring, MD: Center for Operational Oceanographic Products and Services, National Ocean Service, NOAA.
\item Flick, R., K. Knuuti, and S. Gill (2012) Matching mean sea level rise projections to local elevation datums. Journal of Waterway, Port, Coastal, and Ocean Engineering 139(2): 142–146.
\item Breaker, L. C., and A. Ruzmaikin (2013) Estimating rates of acceleration based on the 157-year record of sea level from San Francisco, California, U.S.A. Journal of Coastal Research 29(1): 43–51. doi: \url{http://dx.doi.org/10.2112/JCOASTRES-D-12-00048.1}
\item Church, J. A., P. Woodworth, T. Aarup, and W. S. Wilson (2007) Understanding sea level rise and variability. EOS, Transactions of the American Geophysical Union 88(4): 43.
\item Bindoff, N. L., J. Willebrand, V. Artale, A. Cazenave, J. Gregory, S. Gulev, K. Hanawa, C. Le Quéré, S. Levitus, Y. Nojiri, C. K. Shum, L. D. Talley, and A. Unnikrishnan (2007) Chapter 5, Observations: Oceanic Climate Change and Sea Level. In: Climate Change 2007: The Physical Science Basis. Contribution of Working Group I to the Fourth Assessment Report of the Intergovernmental Panel on Climate Change (S. Solomon, D. Qin, M. Manning, Z. Chen, M. Marquis, K. B. Averyt, M. Tignor, and H. L. Miller, eds.). Cambridge, United Kingdom, and New York, NY: Cambridge University Press. \url{http://www.ipcc.ch/pdf/assessmentreport/ar4/wg1/ar4-wg1-chapter5.pdf}
\item NOAA 2010 National Oceanic and Atmospheric Administration (2010b) Mean Sea Level Trends, San Diego, CA. Center for Operational Oceanographic Products and Services, NOAA. \url{http://tidesandcurrents.noaa.gov/sltrends/sltrends_station.shtml?stnid=9410170}.
\item National Oceanic and Atmospheric Administration, Climate Program Office. \url{http://cpo.noaa.gov/Home/AllNews/TabId/315/ArtMID/668/ArticleID/80/Global-Sea-Level-Rise-Scenarios-for-the-United-States-National-Climate-Assessment.aspx}.
\item Parris, A., P. Bromirski, V. Burkett, D. Cayan, M. Culver, J. Hall, R. Horton, K. Knuuti, R. Moss, J. Obeysekera, A. Sallenger, and J. Weiss (2012) Global Sea Level Rise Scenarios for the U.S. National Climate Assessment. NOAA Technical Report OAR CPO-1. Washington, DC: 31 Dec 13
USACE Climate Change Adaptation Policy Statement, 3 June 2011. \url{http://www.corpsclimate.us/docs/USACEAdaptationPolicy3June2011.pdf}
\item Climate Change Science Program (CCSP) (2009) Synthesis and Assessment Product 4.1: Coastal Sensitivity to Sea level Rise: A Focus on the Mid-Atlantic Region. A report by the U.S. Climate Change Program and the Subcommittee on Global Change Research [J. G. Titus (Coordinating Lead Author), E. K. Anderson, D. Cahoon, S. K. Gill, R. E. Thieler, J. S. Williams (Lead Authors)]. Washington, DC: U.S. Environmental Protection Agency. \url{http://www.climatescience.gov/Library/sap/sap4-1/final-report/default.htm}
\item Intergovernmental Oceanographic Commission, IOC Manual on Sea Level Measurement and Interpretation (1985-2006) Volumes I-IV, \url{http://www.psmsl.org/train_and_info/training/manuals/}
\item Zervas, C.E. (2009) Sea Level Variations of the United States, 1854-2006. National Oceanic and Atmospheric Administration, U.S. Department of Commerce, National Ocean Service, Center for Operational Oceanographic Products and Services, \url{http://www.co-ops.nos.noaa.gov/publications/Tech_rpt_53.pdf}
\end{enumerate}

\end{document}