\documentclass[10pt]{amsart} 
\usepackage{graphicx} 
\usepackage{float} % necessary for placement of figures
\usepackage{amsmath}
\usepackage{enumerate}
\usepackage{tabularx}
\usepackage{gensymb}
\usepackage[style = authoryear, sorting = nyt, backend = biber]{biblatex}
\bibliography{library.bib}

\title{Where to incorporate climate information?}
\author{Andrew Blohm}
\date{\today}

\begin{document}
\maketitle
\section{Case study}
In this section we discuss a potential illustration of the complexities of climate change on infrastructure design.

Through discussions with Aaron Phillips we identified coastal infrastructure as a 
Sea level rise poses significant challenges to the DoD given the large amount of assets and infrastructure deployed along coastlines around the world. 
Another potential case study is master planning more generally.
As a first step, new construction or significant rehab must determine if they exist in a 100 year flood plain.
However, as far as I know these are historical 100-year flood plains and thus do not incorporate sea level rise projections across the lifetime of the project.

One interesting note from the interviews with Aaron and Thad in February at the Pentagon is this notion that Engineers just want the data (just tell them where it is)(i.e., wind loads, temperature and humidity, rainfall amounts, snow loads, etc.).
There assertion was that previously this was static but now its not so where does this information come from.
I think it's important to point out that none of this has been static for quite sometime.
We've been building assets and assuming that risk levels are lower than they actually might be.
Also, there is concern with scenarios and which to use because obviously you need a scenario on future emissions to determine the degree of change.
There are plenty of existing methods for incorporating this uncertainty (i.e., stochastic optimization to name just one). 
Also, for some assets, given the length of life expected, the range of possible futures might be quite small for the variables that it is sensitive to.

The questions become (according to Aaron) how to look at multiple scenarios and look at multiple alternatives; how to apply climate science information; who pays; are previous risk standards okay (i.e,. 1 in 100 year)?
Aaron suggested that executive orders, DODD [not sure what this is]; and Federal Flood Risk Management Standards as the places we could do this.
Finally, this all ties into the fact that much of the important work (i.e., where climate information is most needed) is actually being done by planners (who consult with engineers).

\begin{enumerate}[1.]
\item Select an appropriate UFC. An example might be UFC 415201 Piers and Wharfs, which includes non-specific language on climate change.
\item Select a test location. 
I would recommend reviewing Claudia Tebaldi's paper that includes local sea level rise projections and using something here. 
Another approach is to use John Halls report to identify which sites have the necessary information. 
Finally, look at the email from Aaron Phillips that included planned updates to the UFC. 
The update attempts to include a fudge factor to build high enough to be climate resilient (but this should be tested).  
\item Determine applicable regulations, allowable input data, etc.
FEMA maps are used, AFCE24 for flood resistant design, Base elevation + freeboard (approach doesn't directly account for SLR or Design Flood Elevation); ASC24.14 - flood design standards of the American Society of Civil Engineers.
Not sure if we did this but interview notes suggested we check with Adam Parriss on updated flood maps.
If we take a NAVFAC approach they use Engineering Construction Bulletins (the US Army Corps has their own engineering regulation)
\item Determine local data sources
\item Determine historical, present day, and future return periods under different scenarios (determine FEMA heights, I've found weird discrepancies between their return levels and the ones I calculate)
\item Calculate the return periods for various points in the future 
\item Determine the optimal design of the system (same as UFC used) in each of the time periods
\item If applicable determine the parameters in which the design would continue to function (not optimally but continue to function nonetheless)
\item Determine the optimal investment under this situation accounting for the uncertainty
\item Discuss potential alternatives (i.e., less expensive structures meant to be replaced more frequently?, floating docks, etc.).
Sea level rise poses some unique challenges to docks and wharfs given that ship to pier geometry matters in determining if ships of various sizes can use the facilities.
\end{enumerate}

\section{Random notes}
John Hall stated that Becky's interest in this is from exposure and that it would be a good idea to integrate her interests into our approach.

One organizing framework Richard proposed was:
\section{Current guidelines and policies for floodplain risk management and planning}
National

Existing process in Department or Services
\section{Methods for identifying the flood plain [current and under non-stationarity}
FEMA

Updated FEMA

FEMA + Freeboard

Design flood elevation

Methods under SLR and CC: develop guidance on what information to use under different circumstances (different approaches will have different types and levels of uncertainty, and depending on project characteristics, some approaches will be more appropriate than others)

\section{Policies guiding planning, engineering, and asset management of assets determined to be in the flood plain}
\begin{enumerate}
\item AFCE24 for flood resistant design?
\item 11988 (old flood plain standard)?
\item FFRMF (100 yr plus 2ft)?
\item Note: risk management a consideration. Choice between risk tolerance and cost. Department needs to define acceptable risk.
\item Distinction between risk consequences and likelihood of occurrence. Over vs under protection situations to examine costs/benefits.
\end{enumerate}

\section{Training: for whom, about what}
\begin{enumerate}
\item Tier 1 (greatest leverage): Thad, Aaron, ESEP, others who will expand on current limited guidance and set policy, select what data to use, and provide guidance on how to implement
\item Tier 2: Those using the guidance and updated data: planners and engineers are distinct audiences? 
\item Tier 3: Asset managers? [unclear whether we can go this far but based on conversation could be useful in a future iteration]
\end{enumerate}

\section{Where to incorporate climate information?}
It is an open research question of how to improve the resilience and reduce the vulnerability of human and natural systems to climate change.
In particular, how best to build in climate information? 
Agencies are now required as a result of Executive Order [?] to determine the vulnerability of agency assets to climate change.
For an organization as large as the Department of Defense (DoD) this is a large undertaking. 
In previous work we put forward the idea that conducting vulnerability assessments at each site is inefficient, as the threats of climate change and more importantly the mission impacts are non-homogenously distributed across space.
Nevertheless in this work we didn't identify the optimal means of creating change throughout the organization (i.e., improved resilience, reduced vulnerability).

Who is the target audience?
Who will receive training on climate information?
Tier One higher level that will expand on current limited guidance and set policy, select what data to use, and provide guidance on how to implement.
Tier Two is user groups like planners and engineers; those using the guidance in longer term planning.
Tier Three are asset managers \ldots.

There is a need to move away from deterministic approaches but instead incorporate the uncertainty into decision making processes.
If you are a licensed engineer you need to know what code or practice is. 
If you deviate, you may be liable. Back to changes in professional codes and standards. 
Must keep track of updates.

We have conducted vulnerability analyses across Department of Defense and Department of Energy sites and have begun working towards best practices both in methods for conducting vulnerability analyses and the creation of toolboxes to match with site characteristics.
One issue that always comes up is where to incorporate climate information so as to ensure resilience is maintained for future periods as well.
We find that existing processes are the best spot to incorporate climate information, as compared to generating new processes. 
Existing processes have stakeholders, methods, user communities, etc.
All of which would need to be established for any new process.
The DoD has a number of required products that could benefit from the inclusion of climate projections such as the infrastructure standards through processes such as the Unified Facilities Criteria (UFCs), installation master planning, Integrated Natural Resources Management Plan (INRMP), emergency management planning, etc.\footnote{UFCs are updated on a recurring basis with the frequency of updates dependent upon the application (i.e., building codes every five years versus hyperbaric facilities updated every 20 years)[Conversation with Aaron Philips].}

Who needs the training on and better access to approved climate information?
Much the same as the example of vulnerability analyses, it is inefficient to train all personnel to use and incorporate climate information.
There are a number of positions in the military that don't need access to this training.
How then do we maximize the impact of climate information through the selection of a training community?
What are the training opportunities?

One approach is to update the inputs to some of these longer term documents.
For example, instead of updating individual UFCs we could update the references and data inputs used by the UFCs. 
It was the opinion of Aaron and Thad that the engineers don't need new data instead just point to the data source that they should be using in the process.
UFCs are limited in the degree of change they can engender in that they apply to new construction or major rehabilitation only; in other words they don't apply to asset management.
This suggests a need for asset management policy, in addition. 
The UFCs have a hierarchical structure in which there is a general buildings requirement overarching document.
For each type of structure there are then more specific requirements.
Also, master planning is considered as a separate discipline.
UFCs use commercially based standards to the extent possible.
Standards are also standardized across departments as much as possible.

Department needs to define acceptable risk in an era of nonstationarity.
Historically, has used the 100 year flood plain for coastal sites.
Current policy revolves around whether an asset (or proposed asset) is within the current 100 year flood plain.
If it is then you are required to implement various mitigation measures (i.e., 100 year plus 2 feet). 
If outside, then nothing additional required.
Doesn't take into account non-stationarity.
Big issue is where the engineers should go to get the input data necessary for project design?
This will necessarily tie into scenario selection (though maybe not as much as everyone anticipates given the differences in the scenario values for a parameter of interest in the near to mid term (might not be that much)). 

Climate conditions can affect tests by having parameters that fall outside accepted ranges \parencite{}(David Goade, Aberdeen).
View it as an encroachment factor. 
Help leadership understand how its encroaching. 
Use science, develop better science, improve data collection. 
Dave Goade suggested there would be a possible interest from a combination of three groups: SEG, operations, and meteorology. Idea would be to develop a plan of action and training on improved data collection regarding closures, costs, etc., associated with climate impacts over time. Possible roll out at 2018 training conference, “Sustaining DoD Readiness”. 

Each of these documents and/or processes has an established user community.
At the moment UFCs might make mention of the need to plan or incorporate climate change but lacks sufficient detail on how this should happen.
From our conversations it became apparent that the structural engineers just need guidance. 
Existing planning processes mean that by the time it gets to the design desk it is too late to incorporate climate change \parencite{}[conversation with Aaron Philips].
All considerations need to be incorporated into the planning stage. 
Installation master planning UFC would be a good start but mostly the UFCs are focused on design.
Much of the design process is taking place in the planning process (i.e., sizing, configuration, location, etc.) \parencite{}(Conversation with Aaron Philips).
As a result, one group that needs training are the planners. 
 
Further existing hurdles are exacerbating the problem.
For example, it is presently easier to repair using operations and repair money as compared to replacement using milcon money \parencite{}[conversation with John Hall].

We spent a significant amount of time interviewing stakeholders at sites across the DoD portfolio to better understand existing decision making processes; both in the operation of existing assets and construction of new assets.
We find that many sites have a poor understanding of the impacts of current climate on operations, which limits the ability to incorporate climate projections into decision-making processes.
We suggest a first step be improved data collection at each site on the variables that lead to mission impacts.
For example, we have found in the training and test range community that cancellations of training or test events is tracked but the cause is not.
There is an attribution issue, which means that installation commanders and central command personnel cannot track changes in the ability of these installations to meet mission objectives as a direct or indirect result of changes in climate.
Further, commanders cannot optimize the timing of their training and tests to better reflect the ongoing changes in their climate. 
Finally, not collecting this information prevents the identification of where their own experiences reside in regards to expectations from scenarios (i.e., faster, slower than under scenarios), which impacts the ability to plan.

The structure of the Department of Defense (DoD) would seem to epitomize an organization capable of meeting the challenges associated with developing, communicating, and adopting vulnerability assessments and adaptation alternatives for each of its sites in its large portfolio.
The military is hierarchical with command and control structures, which should be capable of implementing directives from central command at each site. 
However, in our case studies we find that the participatory process might be more important in the DoD than in other organizations because of the challenges that exist.

First, non-civilian members of the DoD are highly transient often changing duty stations.
As a result, the institutional memory is often found in civilian contractors.
The views of these civilian contractors on climate change determines much of what happens next.
Second, existing budget processes do not value projects that reduce vulnerability to climate change.
Third, existing budget processes (especially for large assets) are hard to change once they've begun and thus, even in the case where new information is received it might be impossible to adjust course. 

[Insert discussion of partners and proceses that we worked with]
Existing structures in the test and range communities.
The test range community meets regularly, and reports to a range commanders council.
Engineering Senior Executive Panel (ESEP) reports to Coordinating Panel \parencite{}[conversation with Aaron Philips and John Hall]

\section{Unified Facilities Criteria}
The Unified Facilities Criteria (UFC) provide guidance for developing and maintaining unified facilities design and construction for planning, design, construction, sustainment, restoration, and modernization of DoD facilities \parencite{}(\url{https://www.wbdg.org/ccb/browse_cat.php?c=4}).
The United States Army Corps of Engineers (HQUSACE), Naval Facilities Engineering Command (NAVFAC), and the Air Force Civil Engineer Center (AFCEC) administer the UFC system within their own departments \parencite{} (\url{https://www.wbdg.org/ccb/browse_cat.php?c=4}).
UFCs intended for all participating agencies have a document number that does not end in an alphabetical letter. 
Otherwise, a document with an 'A', 'N', or 'F' at the end of the document corresponds to guidance intended solely for the USACE, NAVFAC, or AFCEC, respectively \parencite{}(\url{https://www.wbdg.org/ccb/browse_cat.php?c=4}).
UFCs are organized into four series: (1) Policy, Procedures, and Guidance; (2) Master Planning; (3) Discipline-specific Criteria; and, (4) Multi-disciplinary and Facility-Specific Design. 
A complete listing of the current UFCs can be found at \url{https://www.wbdg.org/ccb/browse_cat.php?c=4}.

The UFCs are a good program to use as a platform for resilience building for several reasons.
First, UFCs are an existing program and would not require the creation of a new program, training users, and other associated costs of such an effort.
Second, given its use throughout the services, small changes in the UFCs could lead to large changes in resilience.
Third, there are multiple points in the UFC program where climate information could be integrated.
Finally, the program directly ties into the systems that we care about. 

Of the complete set of UFCs, we performed a cursory analysis to identify UFCs that would benefit from the integration of climate information, as well as climate projections.

Series 1: Policy, Procedures, and Guidance
\begin{enumerate}
\item UFC 1-200-02 High Performance and Sustainable Building Requirements, with Change 3 (03-01-2013)
\item UFC 1-201-01 Non-Permanent DoD facilities in Support of Military Operations (01-01-2013)
\item UFC 1-201-02 Assessment of existing facilities for use in Military Operations (06-01-2014)
\item UFC 1-202-01 Host Nation Facilities in Support of Military Operations (09-01-2013)
\item UFC 1-300-07A Design Build Technical Requirements (03-01-2005)
\item UFC 1-300-09N Navy and Marine Corps Design Procedures, with Change 2 (05-01-2014)
\end{enumerate}

Series 2: Master Planning
\begin{enumerate}
\item UFC 2-000-05N (formerly P-80) Facility Planning Criteria for Navy/Marine Corps Shore Installations
\item UFC 2-100-01 Installation Master Planning (05-15-2012)
\end{enumerate}

Series 3: Discipline-specific criteria
\begin{enumerate}
\item UFC 3-201-01 Civil Engineering (-06-01-2013)
\item UFC 3-220-05 Dewatering and Groundwater Control (01-16-2004)
\item UFC 3-320-01 Water storage, Distribution, and Transmission, with Change 2 (11-01-2012)
\item UFC 3-230-03 Water Treatment (11-01-2012)
\item UFC 3-230-06A Subsurface Drainage, with Changes 1-2 (01-16-2004)
\item UFC 3-240-01 Wastewater Collection, with Change 1 (11-01-2012)
\item UFC 3-240-02 Domestic Wastewater Treatment (11-01-2012)
\item UFC 3-260-01 Airfield and Heliport Planning and Design (11-17-2008)
\item FC 3-260-06F Air Force Design, Construction, Maintenance, and Evaluation of Snow and Ice Airfields in Antarctica (06-01-2015)
\item UFC 3-301-01 Structural Engineering, with Change 1 (06-01-2013)
\item UFC 3-400-02 Design: Engineering Weather Data (02-28-2003)
\item UFC 3-401-01 Mechanical Engineering, with Change 1 (07-01-2013)
\item UFC 3-410-01 Heating, Ventilating, and Air Conditioning Systems, with Change 2 (07-01-2013)
\item UFC 3-810-01N Navy and Marine Corps Environmental Engineering for Facility Construction (03-01-2016)
\end{enumerate}

Series 4: Multi-disciplinary and facility-specific design
\begin{enumerate}
\item UFC 4-141-10N Design: Aviation Operation and Support Facilities (01-16-2004)
\item UFC 4-150-02 Dockside Utilities for Ship Service, with Change 5 (05-12-2003)
\item UFC 4-150-06 Military Harbors and Coastal Facilities, with Change 1 (12-12-2001)
\item UFC 4-150-07 Maintenance and Operation: Maintenance of Waterfront Facilities, with Change 1 (06-19-2001)
\item UFC 4-151-10 General Criteria for Waterfront Construction, with Change 1 (09-10-2001) 	
\item UFC 4-152-01 Design: Piers and Wharves; with Change 1 (07-28-2005)
\item UFC 4-152-07 Design: Small Craft Berthing Facilities; with Change 1 (07-14-2009)
\item UFC 4-159-03 Design: Moorings, with Change 1 (10-03-2005)
\item UFC 4-171-01N Design: Aviation Training Facilities (01-16-2004)
\item UFC 4-213-10 Design: Graving Drydocks, with Change 1 (08-15-2002)
\item UFC 4-213-12 Drydocking Facilities Characteristics (06-19-2003)
\end{enumerate}

\section{Guidance documents: Sea Level Rise}
These documents represent the extent of guidance documents for the military services on incorporating sea level rise into existing planning mechanisms, operations, etc.

\begin{enumerate}
\item ER 1100-2-8160: Policies for Referencing Project Elevation Grade to Nationwide Vertical Datums (03-01-2009)
\item ER 1100-2-8162: Incorporating Sea Level Change in Civil Works Program (12-31-2013)
\item ETL 1100-2-1: Procedures to Evaluate Sea Level Change (06-30-2014)
\item ER 1105-2-100: Planning Guidance Notebook (04-2000) \url{http://www.publications.usace.army.mil/USACEPublications/EngineerRegulations/tabid/16441/u43546q/313130352D322D313030/Default.aspx}
\item EM 1110-2-6056: Standards and Procedures for Referencing Project Elevation Grades to Nationwide Vertical Datums. \url{http://www.publications.usace.army.mil/USACEPublications/EngineerManuals/tabid/16439/u43544q/313131302D322D36303536/Default.aspx}
\item ECB 2016-5: Using Non-NOAA Tide Gauge Records for Computing Relative Sea Level Change (01-27-2016)
\item National Research Council (1987) Responding to Changes in Sea Level: Engineering Implications. Washington, DC: National Academy Press. \url{http://www.nap.edu/catalog.php?record_id=1006}
\item National Research Council (2012) Sea-Level Rise for the Coasts of California, Oregon, and Washington: Past, Present, and Future. Committee on Sea Level Rise on California, Oregon, and Washington, Board on Earth Sciences and Resources and Ocean Studies Board. Washington, DC: National Academy Press.
\item Intergovernmental Oceanographic Commission (1985) Manual on Sea Level Measurement and Interpretation, Volume I. Intergovernmental Oceanographic Commission Manuals and Guides-14. \url{http://unesdoc.unesco.org/images/0006/000650/065061eb.pdf}
\item Intergovernmental Oceanographic Commission (2012) Manual on Sea-Level Measurements and Interpretation. Volume 4 — An Update to 2006 (T. Aarup, M. Merrifield, B. Perez, I. Vassie, and P. Woodworth, eds.). IOC Manuals and Guides No. 14, vol. IV; JCOMM Technical Report No. 31; WMO/TD. No. 1339. Paris, France: Intergovernmental Oceanographic Commission.
\item C. Zervas, S. K. Gill, and W. Sweet (2013) Estimating Vertical Land Motion from Long-Term Tide Gauge Records. Technical Report. Silver Spring, MD: Center for Operational Oceanographic Products and Services, National Ocean Service, NOAA.
\item Flick, R., K. Knuuti, and S. Gill (2012) Matching mean sea level rise projections to local elevation datums. Journal of Waterway, Port, Coastal, and Ocean Engineering 139(2): 142–146.
\item Breaker, L. C., and A. Ruzmaikin (2013) Estimating rates of acceleration based on the 157-year record of sea level from San Francisco, California, U.S.A. Journal of Coastal Research 29(1): 43–51. doi: \url{http://dx.doi.org/10.2112/JCOASTRES-D-12-00048.1}
\item Church, J. A., P. Woodworth, T. Aarup, and W. S. Wilson (2007) Understanding sea level rise and variability. EOS, Transactions of the American Geophysical Union 88(4): 43.
\item Bindoff, N. L., J. Willebrand, V. Artale, A. Cazenave, J. Gregory, S. Gulev, K. Hanawa, C. Le Quéré, S. Levitus, Y. Nojiri, C. K. Shum, L. D. Talley, and A. Unnikrishnan (2007) Chapter 5, Observations: Oceanic Climate Change and Sea Level. In: Climate Change 2007: The Physical Science Basis. Contribution of Working Group I to the Fourth Assessment Report of the Intergovernmental Panel on Climate Change (S. Solomon, D. Qin, M. Manning, Z. Chen, M. Marquis, K. B. Averyt, M. Tignor, and H. L. Miller, eds.). Cambridge, United Kingdom, and New York, NY: Cambridge University Press. \url{http://www.ipcc.ch/pdf/assessmentreport/ar4/wg1/ar4-wg1-chapter5.pdf}
\item NOAA 2010 National Oceanic and Atmospheric Administration (2010b) Mean Sea Level Trends, San Diego, CA. Center for Operational Oceanographic Products and Services, NOAA. \url{http://tidesandcurrents.noaa.gov/sltrends/sltrends_station.shtml?stnid=9410170}.
\item National Oceanic and Atmospheric Administration, Climate Program Office. \url{http://cpo.noaa.gov/Home/AllNews/TabId/315/ArtMID/668/ArticleID/80/Global-Sea-Level-Rise-Scenarios-for-the-United-States-National-Climate-Assessment.aspx}.
\item Parris, A., P. Bromirski, V. Burkett, D. Cayan, M. Culver, J. Hall, R. Horton, K. Knuuti, R. Moss, J. Obeysekera, A. Sallenger, and J. Weiss (2012) Global Sea Level Rise Scenarios for the U.S. National Climate Assessment. NOAA Technical Report OAR CPO-1. Washington, DC: 31 Dec 13
USACE Climate Change Adaptation Policy Statement, 3 June 2011. \url{http://www.corpsclimate.us/docs/USACEAdaptationPolicy3June2011.pdf}
\item Climate Change Science Program (CCSP) (2009) Synthesis and Assessment Product 4.1: Coastal Sensitivity to Sea level Rise: A Focus on the Mid-Atlantic Region. A report by the U.S. Climate Change Program and the Subcommittee on Global Change Research [J. G. Titus (Coordinating Lead Author), E. K. Anderson, D. Cahoon, S. K. Gill, R. E. Thieler, J. S. Williams (Lead Authors)]. Washington, DC: U.S. Environmental Protection Agency. \url{http://www.climatescience.gov/Library/sap/sap4-1/final-report/default.htm}
\item Intergovernmental Oceanographic Commission, IOC Manual on Sea Level Measurement and Interpretation (1985-2006) Volumes I-IV, \url{http://www.psmsl.org/train_and_info/training/manuals/}
\item Zervas, C.E. (2009) Sea Level Variations of the United States, 1854-2006. National Oceanic and Atmospheric Administration, U.S. Department of Commerce, National Ocean Service, Center for Operational Oceanographic Products and Services, \url{http://www.co-ops.nos.noaa.gov/publications/Tech_rpt_53.pdf}
\end{enumerate}

\end{document}